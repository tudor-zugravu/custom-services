\section{Introduction}

\subsection{Project Aims, Objectives and Introduction} 

As the concept of entrepreneurship has gained momentum and considering the digital era evolution, more and more business opportunities have emerged, encouraging the appearance of new software based companies. One such affected market is composed of software solutions that promote already existing retail businesses in order to increase their sales and to provide customers with a unified platform for their specific needs. Through such solutions, users are able to identify and select the best available offers considering criteria such as proximity, price and quality. Also, the target vendor segment is broad, encompassing markets varying from restaurants and bars to bicycle repair shops and barber shops.\\

One obstacle that such start-ups are facing is that even though they have identified a specific business opportunity, they are discouraged by the need to develop a software solution. As most of the entrepreneurs do not have advanced knowledge in Software Engineering, they need to turn to software companies for the development of their product, which is often costly.\\

This project aims to develop a platform that provides customisable software products tailored for this category of business ideas. Based on solution related input from the start-up, the system generates a complete software solution that organises and presents a list of offers for the targeted range of products or services in the form of both a web platform, as well as a mobile application. The features of the software solutions cover all of the necessary functionalities looked for in a retail app. The already existing vendors and their products are displayed in a list sorted by distance, price, review mark or category, which can also be presented on an interactive map for better visualisation. The selection of the best offer is aided by descriptions of the products, photos and review marks. Once the customer decides on a certain product, he/she has the option of instant buyout, generating a receipt that can later be redeemed at the specific vendor, following which the experience can be rated. In order to provide directions to the vendor, the customer can either choose to view a map with the shortest route or follow 3D generated images that guide him/her by using the camera through the means of augmented reality. The mobile application also provides notifications features for the current appointments, as well as geolocation notifications that let them know they are in the proximity of a favourite vendor. These are some of the functionalities from which the start-ups can choose for their software solution.\\

The structure of this paper follows the stages underwent for developing a customisable retail solution, starting with an analysis of the market segment and the scoping of the project. After presenting the identified requirements, the planning process is described, followed by outlining and interpreting the system architecture and design. Finally, the implementation approach is related in detail and the final results are interpreted.

\subsection{Background} \label{sub:background}

The retail market segment has been positively affected by the impact of web and mobile applications technologies. Due to the advancements in mobile technologies and the devices miniaturisation, the range of software solutions has expanded, encompassing almost all social domains. Although simplistic in comparison to industrial systems, mobile applications have reached high levels of profitability and allow the centralisation of data on remote platforms, where it is processed and conveyed in a seamless manner to the users.\\

One such existing software solution is the 'Too Good To Go'\cite{too_good_to_go} mobile application, which targets restaurants and taverns that are legally required to throw away excess food at the end of the day, providing a platform through which customers can purchase the leftovers at very low prices. This benefits both the customers and the restaurants by increasing sales, offering cheap food alternatives and reducing the great amount of thrown away products. Even though 'Too Good To Go' targets only unsold food, it has shown great business value, expanding from only a London targeted application to over 10 European cities. The great success of this application is the main incentive for this project's target, this being but one of numerous niche segments that benefit from the design of such a software platform. The rest of the markets will be more easily available using this project's customisable retail system.\\

Their solution consists of both Android and iOS mobile applications, which display the offers either in the form of a table or a map based on proximity and stock. Category based filtering of the restaurants has also been featured in a recent update of their mobile solutions. The customers have the option of instant buyout for any specific offer and can only redeem it at the restaurant between closing hours. Another functionality of this mobile application is the storage of credit card details for more seamless transactions, as well as increased security. In addition, it offers the possibility to rate previous experiences for other customers' future reference. The system does not present many functionalities, as it is focused on the scope and providing a simple and quick mobile solution.\\

The 'Appsmakerstore'\cite{appsmakerstore} system is a web platform that provides the means to create a customisable retail mobile application for both the iOS and the Android mobile operating systems. Even though its does not target vendors based upon specific types of products or services, its relevance derives from the feature of deploying mobile software solutions only by using functionalities filtering systems and a graphical interface for the clients. This is similar to this project's solution, the only differences being the overview design of the generated mobile applications and the list of functionalities that the client can choose from.\\

As the 'Appsmakerstore''s generated applications fall under the retail category, they provide targeted functionalities from which the clients choose for their tailored applications. The list includes displaying the available products, managing the customer's shopping cart, real-time order and requests processing, displaying promotions, photo galleries, push notifications and loyalty systems. In addition, one of the more interesting features is geolocation alert messaging that has the scope to attract customers when they are in the proximity of venues.\\

The system also allows their clients to manage and update the mobile applications' content remotely, further enhancing the possibilities of seamless customisation. However, this feature is possible through the 'Appsmakerstore''s hosting and maintenance package, through which they are the ones who publish the clients' solutions to the AppStore and Google Play. However useful this option is, this project's system aims to generate the software solutions and to offer the clients complete control over their product.