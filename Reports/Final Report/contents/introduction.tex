\section{Introduction}

\subsection{Motivation} 

As the concept of entrepreneurship has gained momentum and considering the digital era evolution, more and more business opportunities have emerged, encouraging the appearance of new software based companies. One such affected market is composed of software solutions that promote already existing retail businesses in order to increase their sales and to provide customers with a unified platform for their specific needs. Through such solutions, users are able to identify and select the best available offers considering criteria such as proximity, price and quality. Also, the target vendor segment is broad, encompassing markets varying from restaurants and bars to bicycle repair shops and barber shops.\\

One obstacle that such start-ups are facing is that even though they have identified a specific business opportunity, they are discouraged by the need to develop a software solution. As most of the entrepreneurs do not have advanced knowledge in Software Engineering, they need to turn to software companies for the development of their product, which is often costly.\\

\subsection{Project Aims and Objectives}

This project aims to develop a platform that provides customisable software products tailored for this category of business ideas. Based on solution related input from the start-up, the system generates a complete software solution that organises and presents a list of offers for the targeted range of products or services in the form of both a web platform, as well as a mobile application. The features of the software solutions cover all of the necessary functionalities looked for in a retail app. The already existing vendors and their products are displayed in a list sorted by distance, price, review mark or category, which can also be presented on an interactive map for better visualisation. The selection of the best offer is aided by descriptions of the products, photos and review marks. Once the customer decides on a certain product, he/she has the option of instant buyout, generating a receipt that can later be redeemed at the specific vendor, following which the experience can be rated. In order to provide directions to the vendor, the customer can either choose to view a map with the shortest route or follow 3D generated images that guide him/her by using the camera through the means of augmented reality. The mobile application also provides notifications features for the current appointments, as well as geolocation notifications that let them know they are in the proximity of a favourite vendor. These are some of the functionalities from which the start-ups can choose for their software solution.\\

The structure of this paper follows the stages underwent for developing a customisable retail solution, starting with an analysis of the market segment and the scoping of the project. After presenting the identified requirements, the planning process is described, followed by outlining and interpreting the system architecture and design. Finally, the implementation approach is related in detail and the final results are interpreted.
