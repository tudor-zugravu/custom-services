\section{Background Theories} 

2.1. Too Good To Go
One such existing software solution is the "Too Good To Go" mobile application. It targets restaurants and bars that are legally required to throw away excess food at the end of the day and provides a platform through which customers can buy the remaining food at very low prices. This benefits both the customers and the restaurants by increasing sales, offering a cheap food alternative and lowering the great amount of food that is thrown away. Even though "Too Good To Go" targets only unsold food, it has shown great business value, expanding from only a London-targeted application to over 10 European cities. The great success of this application is the main incentive for this project's target. This is but one of numerous niche markets that benefit from the design of such a software solution. The rest of the markets will be more easily available using this project's solution.
The "Too Good To Go" consists of a mobile application, both on Android and iOS, which displays the offers either in a list or on a map based on proximity and stock. Filtering of the restaurants based on category has also been added in a recent update to their mobile solutions. The customers have the option to instantly buyout any specific offer and can only redeem it at the restaurant between closing hours. Selection of the favourite eating places is available for the customers. Another functionality of this mobile application is the saving of the credit card details for more seamless future transactions, as well as increased security. In addition, it offers the possibility to rate previous experiences for future reference of other customers. The creators of the system did not add many functionalities, as they focused more on the scope, thus providing a simple and quick mobile solution.
2.2. Appsmakerstore
The "Appsmakerstore" system is a web platform that provides the means to create a customizable retail mobile application for both the iOS and the Android mobile operating systems. Even though it does not target multiple vendors based upon a specific type of product or service, its relevance derives from the feature of deploying mobile software solutions only by using a functionalities filtering system and a graphical interface for the clients. This is very similar to this project's solution, the only differences being the overview design of the generated mobile applications and the list of functionalities that the client can choose from.
As the "Appsmakerstore"'s generated applications fall under the retail category, they provide targeted functionalities that can be included in the client's tailored application. The list includes displaying the available products, managing the customer's shopping cart, real-time order and requests processing, displaying promotions, photo galleries, push notifications and loyalty systems. In addition, one of the more interesting features is GEO auto-PUSH messaging that has the scope to attract customers when they are in the proximity.
The system also allows the main client to manage and update the mobile applications' content remotely, further enhancing the possibilities of seamless customization. However, this is a feature made possible through the "Appsmakerstore"'s hosting and maintenance package, through which they even publish the clients' solutions to the AppStore and Google Play. However useful this option is, this project's system aims to generate the software solutions and to offer the clients complete control over their product.