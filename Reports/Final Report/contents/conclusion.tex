\section{Conclusion}

The main scope of this project was to develop a customisable retail systems generator to be used by entrepreneurs who target specific product or service market segments. The available customisation options cover the larger part of potential target categories. Also, a solution comprised of two different client platforms, a server and a database component is generated in less than five minutes time, thus achieving the success criteria established in the early stages of project planning.\\

The system's business sector has been outlined, and the development process has been described, keeping account of the different stages of the project's lifecycle. The first step was the scoping process, which pointed to the establishment of the success criteria and the system requirements. Afterwards, the planning stage described how the work packages were scheduled for outlining the development sequence. The conditions lead to the identification of the system expected functionalities which, in addition to the abstract business knowledge elements, formed the components of the initial architecture. This structure was further refurbished and improved by using both the Model View Controller and Client Server architectural stereotypes and various relevant design patterns, the result of which laid the basis for the implementation stage. Through this process, the design components evolved from an abstract conceptual form to classes and structures source code. The tools and technologies used for the development are documented thoroughly and frequently used for similar solutions, being chosen for their applicability for the current project. Finally, the testing phase provided the results required for establishing the attainment of the success criteria.\\  

Even though the customisable system's applications possess all the fundamental characteristics of retail solutions, the final product can still be improved and expanded for an even better user experience. Future work could include adding an admin interface for managing the database and the vendor images, developing solutions for all of the major mobile operating systems devices, as well as providing the means of targeting combinations of offer types. Also on the lines of the system's structure, database connection APIs could be defined for the case in which a three tiered architecture is adopted. Another useful addition would be to create database triggers that reset the quantity fields after the location closing hours, thus preparing the system for the next day's purchases. Finally, the next version of the templates could include some form of analytics algorithms that compute statistics based on the recorded transactions and frequently visited information segments of the applications, thus providing also insight into customer behaviour.\\

As a result, the proposed solution is capable of providing competition for the already existing solutions and asserts itself through business logic complexity veiled by a clear user interface. Once the previously stated improvements are integrated, the system will take a complete form and will be distinguishable from a high level of excellence.